%% LyX 2.0.3 created this file.  For more info, see http://www.lyx.org/.
%% Do not edit unless you really know what you are doing.
\documentclass[twoside,english]{paper}
\usepackage{lmodern}
\renewcommand{\ttdefault}{lmodern}
\usepackage[T1]{fontenc}
\usepackage[latin9]{inputenc}
\usepackage[a4paper]{geometry}
\geometry{verbose,tmargin=3cm,bmargin=2.5cm,lmargin=2cm,rmargin=2cm}
\usepackage{color}
\usepackage{babel}
\usepackage{float}
\usepackage{bm}
\usepackage{amsthm}
\usepackage{amsmath}
\usepackage{amssymb}
\usepackage{graphicx}
\usepackage{esint}
\usepackage[unicode=true,pdfusetitle,
 bookmarks=true,bookmarksnumbered=false,bookmarksopen=false,
 breaklinks=false,pdfborder={0 0 0},backref=false,colorlinks=false]
 {hyperref}
\usepackage{breakurl}
\usepackage{mathrsfs}

\makeatletter

%%%%%%%%%%%%%%%%%%%%%%%%%%%%%% LyX specific LaTeX commands.
%% Because html converters don't know tabularnewline
\providecommand{\tabularnewline}{\\}

%%%%%%%%%%%%%%%%%%%%%%%%%%%%%% Textclass specific LaTeX commands.
\numberwithin{equation}{section}
\numberwithin{figure}{section}

%%%%%%%%%%%%%%%%%%%%%%%%%%%%%% User specified LaTeX commands.
\usepackage{babel}

\@ifundefined{showcaptionsetup}{}{%
 \PassOptionsToPackage{caption=false}{subfig}}
\usepackage{subfig}
\makeatother

\begin{document}

\title{Notes on the QED evolution to NLL accuracy}

\maketitle

\begin{abstract}
  In these notes, we discuss the details of the implementation of the
  QED collinear evolution up to next-to-leading logarithmic (NLL)
  accuracy in the variable-flavour-number scheme (VFNS) in the
  presence of both charged leptons and quarks.
\end{abstract}

\tableofcontents{}

\vspace{20pt}

\section{The structure of the DGLAP equation}

Suppose one wants to study the coupled collinear evolution in pure QED
of the photon distribution $\gamma$, the $i$-th down-type quark
distribution with a certain colour ($rgb$)
$d_i\in\{d^r,d^g,d^b, s^r,s^g,s^b, b^r,b^g,b^b\}$, the $k$-th
down-type anti-quark distribution
$\overline{d}_k\in\{\overline{d}^r, \overline{d}^g, \overline{d}^b,
\overline{s}^r, \overline{s}^g, \overline{s}^b, \overline{b}^r,
\overline{b}^g, \overline{b}^b\}$, the $j$-th up-type quark
distribution $u_j\in\{u^r,u^g,u^b, c^r,c^g,c^b, t^r,t^g,t^b\}$, the
$h$-th up-type anti-quark distribution
$\overline{u}_k\in\{\overline{u}^r, \overline{u}^g, \overline{u}^b,
\overline{c}^r, \overline{c}^g, \overline{c}^b, \overline{t}^r,
\overline{t}^g, \overline{t}^b\}$, the $\alpha$-th lepton distribution
$\ell_\alpha \in \{e^-,\mu^-,\tau^-\}$, and the $\beta$-th anti-lepton
distribution $\overline{\ell}_\beta\in\{e^+,\mu^+,\tau^+\}$. The most
general form of the evolution equations reads:
\begin{equation}
\mu^2\frac{\partial}{\partial \mu^2}
\begin{pmatrix}
\ell_\alpha\\
u_j \\
d_i\\
\gamma\\
\overline{d}_k\\
\overline{u}_h\\
\overline{\ell}_\beta
\end{pmatrix} = \sum_{e,l,m,n,\gamma,\delta} 
\begin{pmatrix}
\mathcal{P}_{\ell_\alpha \ell_\gamma} & \mathcal{P}_{\ell_\alpha u_e} & \mathcal{P}_{\ell_\alpha d_l} & \mathcal{P}_{\ell_\alpha \gamma} & \mathcal{P}_{\ell_\alpha \overline{d}_m} & \mathcal{P}_{\ell_\alpha \overline{u}_n} & \mathcal{P}_{\ell_\alpha \overline{\ell}_\delta} \\ 
\mathcal{P}_{u_j\ell_\gamma} & \mathcal{P}_{u_ju_e} & \mathcal{P}_{u_jd_l} & \mathcal{P}_{u_j\gamma} & \mathcal{P}_{u_j\overline{d}_m} & \mathcal{P}_{u_j\overline{u}_n} & \mathcal{P}_{u_j\overline{\ell}_\delta} \\ 
\mathcal{P}_{d_i\ell_\gamma} & \mathcal{P}_{d_iu_e} & \mathcal{P}_{d_id_l} & \mathcal{P}_{d_i\gamma} & \mathcal{P}_{d_i\overline{d}_m} & \mathcal{P}_{d_i\overline{u}_n} & \mathcal{P}_{d_i\overline{\ell}_\delta} \\ 
\mathcal{P}_{\gamma\ell_\gamma} & \mathcal{P}_{\gamma u_e} & \mathcal{P}_{\gamma d_l} & \mathcal{P}_{\gamma\gamma} & \mathcal{P}_{\gamma\overline{d}_m} & \mathcal{P}_{\gamma\overline{u}_n} & \mathcal{P}_{\gamma\overline{\ell}_\delta}\\
\mathcal{P}_{\overline{d}_k\ell_\gamma} & \mathcal{P}_{\overline{d}_ku_e} & \mathcal{P}_{\overline{d}_kd_l} & \mathcal{P}_{\overline{d}_k\gamma} & \mathcal{P}_{\overline{d}_k\overline{d}_m} & \mathcal{P}_{\overline{d}_k\overline{u}_n} & \mathcal{P}_{\overline{d}_k\overline{\ell}_\delta}\\
\mathcal{P}_{\overline{u}_h\ell_\gamma} & \mathcal{P}_{\overline{u}_hu_e} & \mathcal{P}_{\overline{u}_hd_l} & \mathcal{P}_{\overline{u}_h\gamma} & \mathcal{P}_{\overline{u}_h\overline{d}_m} & \mathcal{P}_{\overline{u}_h\overline{u}_n} & \mathcal{P}_{\overline{u}_h\overline{\ell}_\delta}\\
\mathcal{P}_{\overline{\ell}_\beta\ell_\gamma} & \mathcal{P}_{\overline{\ell}_\beta u_e} & \mathcal{P}_{\overline{\ell}_\beta d_l} & \mathcal{P}_{\overline{\ell}_\beta \gamma} & \mathcal{P}_{\overline{\ell}_\beta \overline{d}_m} & \mathcal{P}_{\overline{\ell}_\beta \overline{u}_n} & \mathcal{P}_{\overline{\ell}_\beta \overline{\ell}_\delta}
\end{pmatrix}
\begin{pmatrix}
\ell_\gamma\\
u_e \\
d_l\\
\gamma\\
\overline{d}_m\\
\overline{u}_n\\
\overline{\ell}_\delta
\end{pmatrix}\,,
\label{eq:evolutionsystem}
\end{equation}
where the index $l$ runs over the down-type quarks, the index $m$ over
the down-type anti-quarks, the index $e$ runs over the up-type quarks,
the index $n$ over the up-type anti-quarks, the index $\gamma$ over
the leptons, and the index $\delta$ over the anti-leptons. In
addition, the Mellin convolution integral between the
splitting-function matrix and the vector of distributions in the
r.h.s. is understood. Despite in general all entries of the
splitting-function matrix are different from zero, in massless QED one
can identify the following equalities for the splitting functions that
that involve a photon:
\begin{equation}
  \begin{array}{l}
\mathcal{P}_{\ell_\alpha \gamma} = \mathcal{P}_{\overline{\ell}_\beta
    \gamma}\equiv \mathcal{P}_{\ell \gamma}\,,\\
    \mathcal{P}_{u_j\gamma} = \mathcal{P}_{\overline{u}_h\gamma} \equiv
    \mathcal{P}_{u\gamma}\,,\\
    \mathcal{P}_{d_i\gamma} = \mathcal{P}_{\overline{d}_k\gamma}
    \equiv\mathcal{P}_{d\gamma}\,,
    \end{array}
  \end{equation}
  and:
\begin{equation}
  \begin{array}{l}
    \mathcal{P}_{\gamma\ell_\gamma} =
    \mathcal{P}_{\gamma\overline{\ell}_\delta} \equiv
    \mathcal{P}_{\gamma\ell}\,,\\
    \mathcal{P}_{\gamma u_e} = \mathcal{P}_{\gamma\overline{u}_n} \equiv
    \mathcal{P}_{\gamma u}\,,\\
    \mathcal{P}_{\gamma d_l} = \mathcal{P}_{\gamma\overline{d}_m} \equiv
    \mathcal{P}_{\gamma d}\,.
    \end{array}
\end{equation}
The splitting functions that only involve fermions (quarks and/or
leptons) instead obey the following decompositions:
\begin{equation}
  \begin{array}{l}
    \mathcal{P}_{\ell_\alpha
    \ell_\gamma}=\mathcal{P}_{\overline{\ell}_\alpha
    \overline{\ell}_\gamma} \equiv
    \delta_{\alpha\gamma}\mathcal{P}_{\ell\ell}^V+\mathcal{P}_{\ell\ell}^S \,,\\
    \mathcal{P}_{\overline{\ell}_\beta\ell_\gamma} \equiv
    \mathcal{P}_{\ell_\beta \overline{\ell}_\gamma} =
    \delta_{\beta\gamma}\mathcal{P}_{\ell\overline{\ell}}^V+\mathcal{P}_{\ell\overline{\ell}}^S\,,\\
    \mathcal{P}_{u_ju_e} 
    =\mathcal{P}_{\overline{u}_j\overline{u}_e}\equiv
    \delta_{je}P_{uu}^V+P_{uu}^S
    \,,\\
    \mathcal{P}_{\overline{u}_hu_e} 
    =\mathcal{P}_{u_h\overline{u}_e}\equiv
    \delta_{he}P_{u\overline{u}}^V+P_{u\overline{u}}^S\,,\\
    \mathcal{P}_{d_id_l} 
    =\mathcal{P}_{\overline{d}_i\overline{d}_l}\equiv
    \delta_{il}P_{dd}^V+P_{dd}^S
    \,,\\
    \mathcal{P}_{\overline{d}_kd_l} 
    =\mathcal{P}_{d_k\overline{d}_l}\equiv \delta_{kl}P_{d\overline{d}}^V+P_{d\overline{d}}^S\,,\\

    \mathcal{P}_{\ell_\alpha u_e}= \mathcal{P}_{\overline{\ell}_\alpha \overline{u}_e}\equiv \mathcal{P}_{\ell u}^S\,,\\
    \mathcal{P}_{\ell_\alpha d_l}=\mathcal{P}_{\overline{\ell}_\alpha \overline{d}_l}\equiv \mathcal{P}_{\ell d}^S\,,\\
    \mathcal{P}_{\ell_\alpha \overline{u}_e}= \mathcal{P}_{\overline{\ell}_\alpha u_e}\equiv \mathcal{P}_{\ell \overline{u}}^S\,,\\
    \mathcal{P}_{\ell_\alpha \overline{d}_l}=\mathcal{P}_{\overline{\ell}_\alpha d_l}\equiv \mathcal{P}_{\ell \overline{d}}^S\,,\\

    \mathcal{P}_{u_e\ell_\alpha }= \mathcal{P}_{\overline{u}_e\overline{\ell}_\alpha }\equiv \mathcal{P}_{u\ell }^S\,,\\
    \mathcal{P}_{d_l\ell_\alpha }=\mathcal{P}_{\overline{d}_l\overline{\ell}_\alpha }\equiv \mathcal{P}_{d\ell}^S\,,\\
    \mathcal{P}_{\overline{u}_e\ell_\alpha }= \mathcal{P}_{ u_e \overline{\ell}_\alpha}\equiv \mathcal{P}_{\overline{u}\ell }^S\,,\\
    \mathcal{P}_{\overline{d}_l\ell_\alpha }=\mathcal{P}_{d_l \overline{\ell}_\alpha }\equiv \mathcal{P}_{ \overline{d}\ell}^S\,,\\
    \mathcal{P}_{u_l d_k} = \mathcal{P}_{d_l u_k}=\mathcal{P}_{\overline{u}_l \overline{d}_k} =\mathcal{P}_{\overline{d}_l \overline{u}_k} \equiv \mathcal{P}_{ud}^S\,,\\
    \mathcal{P}_{u_l \overline{d}_k} = \mathcal{P}_{\overline{d}_l u_k}=\mathcal{P}_{\overline{u}_l {d}_k} =\mathcal{P}_{{d}_l \overline{u}_k} \equiv \mathcal{P}_{u\overline{d}}^S\,.\\
    \end{array}
\end{equation}

Each one of splitting functions on the r.h.s. of the $\equiv$ symbol
above admits a perturbative expansion that truncated to
next-to-leading order (as appropriate to achieve NLL accuracy at the
level of the solution of the evolution equation), reads:
\begin{equation}
\mathcal{P}=\left(\frac{\alpha}{4\pi}\right)\mathcal{P}^{(0)}+\left(\frac{\alpha}{4\pi}\right)^2\mathcal{P}^{(1)}\,.
\end{equation}
The coefficients $\mathcal{P}^{(0)}$ and $\mathcal{P}^{(1)}$ undergo
further simplifications. Let us start with the leading order one
$\mathcal{P}^{(0)}$. At this order all of the pure-singlet
contributions vanish, that is $\mathcal{P}_{xy}^{(0),S} = 0$ for all
pairs $xy$, and so do the valence fermion/anti-fermion splitting
functions vanish, \textit{i.e.}
$\mathcal{P}_{\ell\overline{\ell}}^{(0),V}=\mathcal{P}_{u\overline{u}}^{(0),V}=\mathcal{P}_{d\overline{d}}^{(0),V}=0$.
In addition, the splitting functions that survive obey the following
equalities:
\begin{equation}
  \begin{array}{l}
  \mathcal{P}_{\ell \gamma}^{(0)} = e_\ell^2 P_{f\gamma}^{(0)}\,,\\
  \mathcal{P}_{u\gamma}^{(0)} = e_u^2P_{f\gamma}^{(0)}\,,\\
    \mathcal{P}_{d\gamma}^{(0)} = e_d^2P_{f\gamma}^{(0)}\,,\\
    \\
  \mathcal{P}_{\gamma \ell }^{(0)} = e_\ell^2 P_{\gamma f}^{(0)}\,,\\
  \mathcal{P}_{\gamma u}^{(0)} = e_u^2P_{\gamma f}^{(0)}\,,\\
    \mathcal{P}_{\gamma d}^{(0)} = e_d^2P_{\gamma f}^{(0)}\,,\\
    \\
    \mathcal{P}_{\ell \ell}^{V,(0)} = e_\ell^2 P_{ff}^{V,(0)}\,,\\
    \mathcal{P}_{uu}^{V,(0)} = e_u^2P_{ff}^{V,(0)}\,,\\
          \mathcal{P}_{dd}^{V,(0)} = e_d^2 P_{ff}^{V,(0)}\,,\\
    \end{array}
\end{equation}
where $e_\ell^2=1$, $e_u^2=4/9$, and $e_d^2=1/0$ are the electric charges of up- and
down-type quarks, respectively.

Let us now move to the next-to-leading order coefficient
$\mathcal{P}^{(1)}$. At this order, the pure-single splitting
functions are different from zero but all the same up to an overall
factor:
\begin{equation}
  \mathcal{P}_{xy}^{S,(1)} = e_x^2e_y^2P_{ff}^{S,(1)}
\end{equation}
where $x$ and $y$ are either a lepton, or a down-type quark, or an
up-type quark.  For the other splitting functions one finds:
\begin{equation}
  \begin{array}{l}
    \mathcal{P}_{\ell \gamma}^{(1)} = e_\ell^4 P_{f\gamma}^{(1)}\,,\\
    \mathcal{P}_{u\gamma}^{(1)} = e_u^4P_{f\gamma}^{(1)}\,,\\
    \mathcal{P}_{d\gamma}^{(1)} = e_d^4P_{f\gamma}^{(1)}\,,\\
    \\
    \mathcal{P}_{\gamma \ell }^{(1)} = e_\ell^4 P_{\gamma f}^{(1)}+e_\ell^2 P_{\gamma f ,n}^{(1)}\,,\\
    \mathcal{P}_{\gamma u}^{(1)} = e_u^4P_{\gamma f}^{(1)}+e_u^2 P_{\gamma f ,n}^{(1)}\,,\\
    \mathcal{P}_{\gamma d}^{(1)} = e_d^4P_{\gamma f}^{(1)}+e_d^2 P_{\gamma f ,n}^{(1)}\,,\\
    \\
    \mathcal{P}_{\ell \ell}^{V,(1)} = e_\ell^4 P_{ff}^{V,(1)}+e_\ell^2 P_{ff,n}^{V,(1)}\,,\\
    \mathcal{P}_{uu}^{V,(1)} = e_u^4P_{ff}^{V,(1)}+e_u^2P_{ff,n}^{V,(1)}\,,\\
    \mathcal{P}_{dd}^{V,(1)} = e_d^4 P_{ff}^{V,(1)}+e_d^2P_{ff,n}^{V,(1)}\,,\\
    \\
    \mathcal{P}_{\ell\overline{\ell}}^{V,(1)} = e_\ell^4 P_{f\overline{f}}^{V,(1)}\,,\\
    \mathcal{P}_{u\overline{u}}^{V,(1)} = e_u^4P_{f\overline{f}}^{V,(1)}\,,\\
    \mathcal{P}_{d\overline{d}}^{V,(1)} = e_d^4 P_{f\overline{f}}^{V,(1)}\,.
    \end{array}
\end{equation}

We are now in a position to rewrite the evolution system in
Eq.~(\ref{eq:evolutionsystem}) exploiting the equalities identified
above:
%%%%%%%%%%%%%%%%%%%%%%%%%%%%%%%%%%
\begin{equation}
      \tiny
  \begin{array}{l}
\displaystyle \mu^2\frac{\partial}{\partial \mu^2}
\begin{pmatrix}
\ell_\alpha^+\\
u_j^+ \\
d_i^+\\
\gamma
\end{pmatrix} =\displaystyle \left[\left(\frac{\alpha}{4\pi}\right)
\begin{pmatrix}
e_\ell^2P_{ff}^{V,(0)} & 0 & 0 & 2e_\ell^2{P}_{f\gamma}^{(0)}\\ 
0 & e_u^2P_{ff}^{V,(0)} & 0 & 2e_u^2{P}_{f\gamma}^{(0)}\\ 
0 & 0 & e_d^2P_{ff}^{V,(0)} & 2e_d^2{P}_{f\gamma}^{(0)}\\ 
e_\ell^2{P}_{\gamma f}^{(0)} & e_u^2{P}_{\gamma f}^{(0)} & e_d^2{P}_{\gamma f}^{(0)} & {P}_{\gamma\gamma}^{(0)}
\end{pmatrix}
                                           \begin{pmatrix}
                                             \ell_\alpha^+\\
                                             u_j^+ \\
                                             d_i^+\\
                                             \gamma\\
                                           \end{pmatrix}
\right.\\
    \\
              +\displaystyle\left(\frac{\alpha}{4\pi}\right)^2
\begin{pmatrix}
e_\ell^4P_{ff}^{V,(1)}+e_\ell^2P_{ff,n}^{V,(1)}+e_\ell^4P_{f\overline{f}}^{V,(1)} & 0 & 0 & 2e_\ell^4{P}_{f\gamma}^{(1)}\\ 
0 & e_u^4P_{ff}^{V,(1)}+e_u^2P_{ff,n}^{V,(1)}+e_u^4P_{f\overline{f}}^{V,(1)} & 0 & 2e_u^4{P}_{f\gamma}^{(1)}\\ 
0 & 0 & e_d^4P_{ff}^{V,(1)}+e_d^2P_{ff,n}^{V,(1)}+e_d^4P_{f\overline{f}}^{V,(1)} & 2e_d^4{P}_{f\gamma}^{(1)}\\ 
e_\ell^4{P}_{\gamma f}^{(1)} +e_\ell^2 P_{\gamma f ,n}^{(1)}& e_u^4{P}_{\gamma f}^{(1)}+e_u^2 P_{\gamma f ,n}^{(1)} & e_d^4{P}_{\gamma f}^{(1)} +e_d^2 P_{\gamma f ,n}^{(1)}& {P}_{\gamma\gamma}^{(1)}
\end{pmatrix}
                                                                   \begin{pmatrix}
                                                                     \ell_\alpha^+\\
                                                                     u_j^+ \\
                                                                     d_i^+\\
                                                                     \gamma
                                                                   \end{pmatrix}\\
    \\
                  \left. +\displaystyle\left(\frac{\alpha}{4\pi}\right)^2
                  2P_{ff}^{S,(1)}\begin{pmatrix}
                     e_\ell^4 & e_\ell^2e_u^2 & e_\ell^2e_d^2 & 0 \\
                     e_u^2e_\ell^2 & e_u^4 & e_u^2e_d^2 & 0 \\
                    e_d^2e_\ell^2 & e_d^2e_u^2 & e_d^4 & 0 \\
                    0 & 0 & 0 & 0
                  \end{pmatrix}
\begin{pmatrix}
\Sigma_\ell\\
\Sigma_u \\
\Sigma_d\\
\gamma
\end{pmatrix} 
\right]
\,,
\end{array}
\end{equation}





%%%%%%%%%%%%%%%%%%%%%%%%%%%%%%%%%%
\begin{equation}
      \tiny
  \begin{array}{l}
\displaystyle \mu^2\frac{\partial}{\partial \mu^2}
\begin{pmatrix}
d_i^-\\
u_j^-\\
\ell_\alpha^-
\end{pmatrix} =\displaystyle \left[\left(\frac{\alpha}{4\pi}\right)
\begin{pmatrix}
e_d^2{P}_{ff}^{V,(0)} & 0 & 0\\
0 & e_u^2{P}_{ff}^{V,(0)} & 0\\
0 & 0 & e_\ell^2{P}_{ff}^{V,(0)}
\end{pmatrix}
                                           \begin{pmatrix}
                                             d_i^-\\
                                             u_j^-\\
                                             \ell_\alpha^-
                                           \end{pmatrix}
\right.\\
    \\
\left.               +\displaystyle\left(\frac{\alpha}{4\pi}\right)^2
\begin{pmatrix}
e_d^4P_{ff}^{V,(1)}+e_d^2P_{ff,n}^{V,(1)}-e_d^4P_{f\overline{f}}^{V,(1)} & 0 & 0\\
0 & e_u^4P_{ff}^{V,(1)}+e_u^2P_{ff,n}^{V,(1)}-e_u^4P_{f\overline{f}}^{V,(1)} & 0\\
0 & 0 & e_\ell^4P_{ff}^{V,(1)}+e_\ell^2P_{ff,n}^{V,(1)}-e_\ell^4P_{f\overline{f}}^{V,(1)}
\end{pmatrix}
                                           \begin{pmatrix}
                                             d_i^-\\
                                             u_j^-\\
                                             \ell_\alpha^-
                                           \end{pmatrix}
        \right]
\,,
\end{array}
\end{equation}
where we have used the following definitions:
\begin{equation}
  \begin{array}{l}
    \ell_\alpha^{\pm} = \ell_\alpha \pm \overline{\ell}_\alpha\\
    u_j^{\pm} = u_j \pm \overline{u}_j\\
    d_i^{\pm} = d_i \pm \overline{d}_i\,
    \end{array}
\end{equation}
and also introduced:
\begin{equation}
\begin{array}{lll}
\displaystyle \Sigma_\ell = \sum_{\alpha=e,\mu,\tau} \ell_{\alpha}^+&\quad \displaystyle\Sigma_u = \sum_{k=i}^{N_c n_u} u_k^+ & \qquad \displaystyle \Sigma_d
= \sum_{k=i}^{N_c n_d} d_k^+ \,.
\end{array}
\end{equation}
For completeness we also introduce:
\begin{equation}
\begin{array}{lll}
\displaystyle  V_\ell = \sum_{\alpha=e,\mu,\tau} \ell_{\alpha}^-&\quad\displaystyle V_u = \sum_{k=i}^{N_c n_u} u_k^- & \qquad \displaystyle V_d
= \sum_{k=i}^{N_c n_d} d_k^-\,.
\end{array}
\end{equation}
Notice that quarks are summed also over their colour degree of freedom.

In order to diagonalise as much as possible the evolution matrix in
the presence of QED corrections avoiding unnecessary couplings between
parton distributions, we propose the following evolution basis:
\begin{equation}
\begin{array}{ll}
  \mbox{1) }\gamma & \\
  \mbox{2) }\displaystyle \Sigma_\ell & \quad \mbox{11) }\displaystyle V_\ell\\
  \mbox{3) }\displaystyle \Sigma_u & \quad \mbox{12) }\displaystyle V_u\\
  \mbox{4) }\displaystyle \Sigma_d & \quad \mbox{13) }\displaystyle V_d\\
  \mbox{5) }\displaystyle  T_{1}^{\ell} = \ell_e^+ - \ell_\mu^+&\quad \mbox{14) }\displaystyle  T_{2}^{\ell} = \ell_e^+ + \ell_\mu^+-2\ell_\tau^+\\
  \mbox{6) }\displaystyle  V_{1}^{\ell} = \ell_e^- - \ell_\mu^-&\quad \mbox{15) }\displaystyle  V_{2}^{\ell} = \ell_e^- + \ell_\mu^--2\ell_\tau^-\\
  \mbox{7) }T_1^u = u^+ - c^+ &\quad \mbox{16) }V_1^u = u^- - c^- \\
  \mbox{8) }T_2^u = u^+ + c^+ - 2t^+ &\quad \mbox{17) }V_2^u = u^- + c^- - 2t^-\\
  \mbox{9) }T_1^d = d^+ - s^+ &\quad \mbox{18) }V_1^d = d^- - s^- \\
  \mbox{10) }T_2^d = d^+ + s^+ - 2b^+ &\quad \mbox{19) }V_2^d = d^- + s^- - 2b^-
\end{array}
\label{eq:EvolutionBasis}
\end{equation}
we the introduce the following splitting functions:
\begin{equation}
\begin{array}{l}
\displaystyle P_p^{\pm} = \left(\frac{\alpha}{4\pi}\right)
e_p^2P_{ff}^{V,(0)}+\left(\frac{\alpha}{4\pi}\right)^2\left[e_p^4P_{ff}^{V,(1)}+e_p^2P_{ff,n}^{V,(1)}\pm
e_p^4P_{f\overline{f}}^{V,(1)}\right]\\
\\
\displaystyle P_{p\gamma}=\left(\frac{\alpha}{4\pi}\right)
2N_c^{(p)}n_p e_p^2P_{f\gamma}^{V,(0)}+\left(\frac{\alpha}{4\pi}\right)^2 2N_c^{(p)}n_p e_p^4P_{f\gamma}^{V,(1)}\\
\\
\displaystyle P_{\gamma p}=\left(\frac{\alpha}{4\pi}\right)
e_p^2P_{\gamma
  f}^{V,(0)}+\left(\frac{\alpha}{4\pi}\right)^2\left[e_p^4P_{\gamma
  f}^{V,(1)}+e_p^2P_{\gamma f,n}^{V,(1)}\right]\\
\\
\displaystyle
P_{pp'}^{\rm PS} = \left(\frac{\alpha}{4\pi}\right)^22N_c^{(p)}n_pe_p^2e_{p'}^2P_{ff}^{S,(1)}\\
\\
\displaystyle
P_{pp} = \displaystyle
  P_p^{+}+P_{pp}^{\rm PS}\\
\\
\displaystyle P_{\gamma \gamma}=\left(\frac{\alpha}{4\pi}\right)
P_{\gamma \gamma}^{V,(0)}+\left(\frac{\alpha}{4\pi}\right)^2P_{\gamma\gamma}^{V,(1)}
\end{array}
\end{equation}
with $p,p'=\ell,u,d$, and $N_c^{(\ell)}=1$ and
$N_c^{(u)}=N_c^{(d)}=N_c=3$.
With these definitions at hand we can finally write the full set of
systems of evolution equation using the basis given in Eq.~(\ref{eq:EvolutionBasis}):
\begin{equation}
\mu^2\frac{\partial}{\partial \mu^2}
\begin{pmatrix}
\gamma\\
\Sigma_\ell\\
\Sigma_u\\
\Sigma_d
\end{pmatrix} =
\begin{pmatrix}
P_{\gamma\gamma} & P_{\gamma \ell} & P_{\gamma u}& P_{\gamma d}\\
P_{\ell \gamma} & {P}_{\ell\ell} & P_{\ell u}^{\rm PS}& P_{\ell d}^{\rm PS}\\
P_{u \gamma} & P_{u\ell}^{\rm PS} & {P}_{uu} & P_{ud}^{\rm PS} \\
P_{d \gamma} & P_{d\ell}^{\rm PS} & P_{du}^{\rm PS} & {P}_{dd}
\end{pmatrix}
\begin{pmatrix}
\gamma\\
\Sigma_\ell\\
\Sigma_u\\
\Sigma_d
\end{pmatrix}
\label{eq:singletevol}
\end{equation}
\begin{equation}
\mu^2\frac{\partial V_{p}}{\partial \mu^2} = P_p^- V_{p}
\label{eq:valenceevol}
\end{equation}
\begin{equation}
\mu^2\frac{\partial T_{1,2}^{p}}{\partial \mu^2} = P_p^+ T_{1,2}^{p}
\label{eq:Tevol}
\end{equation}
\begin{equation}
\mu^2\frac{\partial V_{1,2}^{p}}{\partial \mu^2} = P_p^- T_{1,2}^{p}
\label{eq:Vevol}
\end{equation}

\section{Threshold crossing}

In order to implement the evolution in the variable-flavour-number
scheme (VFNS), it is necessary to introduce thresholds. Thresholds
need to be ordered in value. To do so, we use the PDG mass values:
\begin{equation}
\begin{array}{l}
m_e = 0.5109989461\cdot 10^{-3} \mbox{ GeV}\\ 
m_u = 2.16\cdot 10^{-3} \mbox{ GeV}\\
m_d = 4.67\cdot 10^{-3} \mbox{ GeV}\\ 
m_s = 0.093 \mbox{ GeV}\\
m_\mu = 0.1056583745 \mbox{ GeV}\\
m_c = 1.27\mbox{ GeV}\\
m_\tau = 1.77686 \mbox{ GeV}\\
m_b = 4.18 \mbox{ GeV}\\
m_t = 172.76 \mbox{ GeV}
\end{array}
\end{equation}
Assuming this order, the number of active leptons $n_\ell$, dow-type
quark $n_d$ and up-type quarks $n_u$ changes as follows as the scale
increases. In addition, while Eq.~(\ref{eq:singletevol})
and~(\ref{eq:valenceevol}) are valid as they are at all scales, the
$T_{1,2}^{p}$ and $V_{1,2}^{p}$ evolves as in Eqs.~(\ref{eq:Tevol})
and ~(\ref{eq:Vevol}) only for $n_p=3$. Assuming no intrinsic
contributions, they reduce to the corresponding singlet $\Sigma_{p}$
or total valence $V_{p}$ distributions:
\begin{itemize}
\item $Q<m_e$:
  \begin{itemize}
  \item $n_\ell=0$, $n_d=0$, $n_u=0$
  \item $T_1^\ell = T_2^\ell = \Sigma_\ell$
  \item $T_1^d = T_2^d = \Sigma_d$
  \item $T_1^u = T_2^u = \Sigma_u$
  \end{itemize}
\item $m_e\leq Q < m_u$:
  \begin{itemize}
  \item $n_\ell=1$, $n_d=0$, $n_u=0$
  \item $T_1^\ell = T_2^\ell = \Sigma_\ell$
  \item $T_1^d = T_2^d = \Sigma_d$
  \item $T_1^u = T_2^u = \Sigma_u$
  \end{itemize}
\item $m_u\leq Q < m_d$:
  \begin{itemize}
  \item $n_\ell=1$, $n_d=0$, $n_u=1$
  \item $T_1^\ell = T_2^\ell = \Sigma_\ell$
  \item $T_1^d = T_2^d = \Sigma_d$
  \item $T_1^u = T_2^u = \Sigma_u$
  \end{itemize}
\item $m_d\leq Q < m_s$:
  \begin{itemize}
  \item $n_\ell=1$, $n_d=1$, $n_u=1$
  \item $T_1^\ell = T_2^\ell = \Sigma_\ell$
  \item $T_1^d = T_2^d = \Sigma_d$
  \item $T_1^u = T_2^u = \Sigma_u$
  \end{itemize}
\item $m_s\leq Q < m_\mu$:
  \begin{itemize}
  \item $n_\ell=1$, $n_d=2$, $n_u=1$
  \item $T_1^\ell = T_2^\ell = \Sigma_\ell$
  \item $T_2^d = \Sigma_d$
  \item $T_1^u = T_2^u = \Sigma_u$
  \end{itemize}
\item $m_\mu\leq Q < m_c$:
  \begin{itemize}
  \item $n_\ell=2$, $n_d=2$, $n_u=1$
  \item $T_2^\ell = \Sigma_\ell$
  \item $T_2^d = \Sigma_d$
  \item $T_1^u = T_2^u = \Sigma_u$
  \end{itemize}
\item $m_c\leq Q < m_\tau$:
  \begin{itemize}
  \item $n_\ell=2$, $n_d=2$, $n_u=2$
  \item $T_2^\ell = \Sigma_\ell$
  \item $T_2^d = \Sigma_d$
  \item $T_2^u = \Sigma_u$
  \end{itemize}
\item $m_\tau\leq Q < m_b$:
  \begin{itemize}
  \item $n_\ell=3$, $n_d=2$, $n_u=2$
  \item $T_2^d = \Sigma_d$
  \item $T_2^u = \Sigma_u$
  \end{itemize}
\item $m_b\leq Q < m_t$:
  \begin{itemize}
  \item $n_\ell=3$, $n_d=3$, $n_u=2$
  \item $T_2^u = \Sigma_u$
  \end{itemize}
\item $Q \geq m_t$:
  \begin{itemize}
  \item $n_\ell=3$, $n_d=3$, $n_u=3$
  \end{itemize}
\end{itemize}

\section{The $\Delta$ scheme}

In order to implemet the $\Delta$ scheme, we need to apply a
tranformation to the matrix of evolution kernels. The transformed
system of evolution equations reads:
\begin{equation}
\mu^2\frac{d}{d\mu^2}\mathbf{f} =
\beta(a(\mu))\mathbb{J}(1+a(\mu)\mathbb{J})^{-1} + (1+a(\mu)\mathbb{J})\mathcal{P}(a(\mu))(1+a(\mu)\mathbb{J})^{-1}\mathbf{f}\,,
\end{equation}
where $\mathbf{f}$ is the vector of PDFs, $\mathcal{P}$ the matrix of
splitting functions, and $\mathbb{J}$ the matrix that parametrises the
transformation from the $\overline{\mbox{MS}}$ to the $\Delta$
scheme. Since the evolution is effectivelly written as differential in
$a$, one finds that:
\begin{equation}
\frac{d}{da}\mathbf{f} =
\mathbb{J}(1+a(\mu)\mathbb{J})^{-1} + (1+a(\mu)\mathbb{J})\frac{\mathcal{P}(a(\mu))}{\beta(a(\mu))}(1+a(\mu)\mathbb{J})^{-1}\mathbf{f}\,.
\end{equation}
Now we need to specify the form of the matrix $\mathbb{J}$. In the
evolution basis in Eq.~(\ref{eq:evolutionsystem}), the explicit form
of $\mathbb{J}$ is:
\begin{equation}
\mathbb{J} =
\begin{pmatrix}
\delta_{\ell_\alpha e^-} \delta_{\ell_\gamma e^-}J_{\ell\ell}& 0 & 0 & 0 & 0 & 0 & 0\\
 0 & 0 & 0 & 0 & 0 & 0 & 0\\
 0 & 0 & 0 & 0 & 0 & 0 & 0\\
\delta_{\ell_\gamma e^-}J_{\gamma\ell} & 0 & 0 & 0 & 0 & 0 & \delta_{\ell_\delta e^+}J_{\gamma\ell}\\
 0 & 0 & 0 & 0 & 0 & 0 & 0\\
 0 & 0 & 0 & 0 & 0 & 0 & 0\\
 0 & 0 & 0 & 0 & 0 & 0 & \delta_{\ell_\beta e^+} \delta_{\ell_\delta e^+}J_{\ell\ell}
\end{pmatrix}\,.
\end{equation}
Finally, in the $\Delta$ scheme, the system of evolution equations
takes the form:
\begin{equation}
\frac{\partial}{\partial a}
\begin{pmatrix}
\gamma\\
\Sigma_\ell\\
\Sigma_u\\
\Sigma_d
\end{pmatrix} =\left[\begin{pmatrix}
0 &\frac{ J_{\gamma \ell}}{1+a J_{\ell\ell}} & 0&0\\
0 &\frac{ J_{\ell\ell}}{1+a J_{\ell\ell}} & 0 & 0 \\
0 &0 & 0 & 0 \\
0 &0 & 0 & 0
\end{pmatrix}
+\frac{1}{\beta}
\begin{pmatrix}
1 & aJ_{\gamma \ell} & 0&0\\
0 & 1+aJ_{\ell\ell} & 0 & 0 \\
0 &0 & 1 & 0 \\
0 &0 & 0 & 1
\end{pmatrix}\begin{pmatrix}
P_{\gamma\gamma} & P_{\gamma \ell} & P_{\gamma u}& P_{\gamma d}\\
P_{\ell \gamma} & {P}_{\ell\ell} & P_{\ell u}^{\rm PS}& P_{\ell d}^{\rm PS}\\
P_{u \gamma} & P_{u\ell}^{\rm PS} & {P}_{uu} & P_{ud}^{\rm PS} \\
P_{d \gamma} & P_{d\ell}^{\rm PS} & P_{du}^{\rm PS} & {P}_{dd}
\end{pmatrix}\begin{pmatrix}
1 & -\frac{aJ_{\gamma \ell}}{1+a J_{\ell\ell}} & 0&0\\
0 & \frac{1}{1+aJ_{\ell\ell}} & 0 & 0 \\
0 &0 & 1 & 0 \\
0 &0 & 0 & 1
\end{pmatrix}\right]
\begin{pmatrix}
\gamma\\
\Sigma_\ell\\
\Sigma_u\\
\Sigma_d
\end{pmatrix}
\label{eq:singletevolDelta}
\end{equation}
\begin{equation}
\frac{\partial V_{p}}{\partial a} = \left[\frac{J_{\ell\ell}}{1+aJ_{\ell\ell}}+\frac{P_p^-}{\beta} \right] V_{p}
\label{eq:valenceevolDelta}
\end{equation}
\begin{equation}
\frac{\partial T_{1,2}^{p}}{\partial a} = \left[\frac{J_{\ell\ell}}{1+aJ_{\ell\ell}}+\frac{P_p^+}{\beta} \right] T_{1,2}^{p}
\label{eq:TevolDelta}
\end{equation}
\begin{equation}
\frac{\partial V_{1,2}^{p}}{\partial a} = \left[\frac{J_{\ell\ell}}{1+aJ_{\ell\ell}}+\frac{P_p^-}{\beta} \right]T_{1,2}^{p}
\label{eq:VevolDelta}
\end{equation}


\end{document}

